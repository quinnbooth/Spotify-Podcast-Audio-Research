% Options for packages loaded elsewhere
\PassOptionsToPackage{unicode}{hyperref}
\PassOptionsToPackage{hyphens}{url}
%
\documentclass[
]{article}
\usepackage{amsmath,amssymb}
\usepackage{lmodern}
\usepackage{iftex}
\ifPDFTeX
  \usepackage[T1]{fontenc}
  \usepackage[utf8]{inputenc}
  \usepackage{textcomp} % provide euro and other symbols
\else % if luatex or xetex
  \usepackage{unicode-math}
  \defaultfontfeatures{Scale=MatchLowercase}
  \defaultfontfeatures[\rmfamily]{Ligatures=TeX,Scale=1}
\fi
% Use upquote if available, for straight quotes in verbatim environments
\IfFileExists{upquote.sty}{\usepackage{upquote}}{}
\IfFileExists{microtype.sty}{% use microtype if available
  \usepackage[]{microtype}
  \UseMicrotypeSet[protrusion]{basicmath} % disable protrusion for tt fonts
}{}
\makeatletter
\@ifundefined{KOMAClassName}{% if non-KOMA class
  \IfFileExists{parskip.sty}{%
    \usepackage{parskip}
  }{% else
    \setlength{\parindent}{0pt}
    \setlength{\parskip}{6pt plus 2pt minus 1pt}}
}{% if KOMA class
  \KOMAoptions{parskip=half}}
\makeatother
\usepackage{xcolor}
\IfFileExists{xurl.sty}{\usepackage{xurl}}{} % add URL line breaks if available
\IfFileExists{bookmark.sty}{\usepackage{bookmark}}{\usepackage{hyperref}}
\hypersetup{
  pdftitle={Spotify Podcasts vs.~Audio Features: Statistical Testing},
  pdfauthor={Lara Karacasu \& Quinn Booth},
  hidelinks,
  pdfcreator={LaTeX via pandoc}}
\urlstyle{same} % disable monospaced font for URLs
\usepackage[margin=1in]{geometry}
\usepackage{color}
\usepackage{fancyvrb}
\newcommand{\VerbBar}{|}
\newcommand{\VERB}{\Verb[commandchars=\\\{\}]}
\DefineVerbatimEnvironment{Highlighting}{Verbatim}{commandchars=\\\{\}}
% Add ',fontsize=\small' for more characters per line
\usepackage{framed}
\definecolor{shadecolor}{RGB}{248,248,248}
\newenvironment{Shaded}{\begin{snugshade}}{\end{snugshade}}
\newcommand{\AlertTok}[1]{\textcolor[rgb]{0.94,0.16,0.16}{#1}}
\newcommand{\AnnotationTok}[1]{\textcolor[rgb]{0.56,0.35,0.01}{\textbf{\textit{#1}}}}
\newcommand{\AttributeTok}[1]{\textcolor[rgb]{0.77,0.63,0.00}{#1}}
\newcommand{\BaseNTok}[1]{\textcolor[rgb]{0.00,0.00,0.81}{#1}}
\newcommand{\BuiltInTok}[1]{#1}
\newcommand{\CharTok}[1]{\textcolor[rgb]{0.31,0.60,0.02}{#1}}
\newcommand{\CommentTok}[1]{\textcolor[rgb]{0.56,0.35,0.01}{\textit{#1}}}
\newcommand{\CommentVarTok}[1]{\textcolor[rgb]{0.56,0.35,0.01}{\textbf{\textit{#1}}}}
\newcommand{\ConstantTok}[1]{\textcolor[rgb]{0.00,0.00,0.00}{#1}}
\newcommand{\ControlFlowTok}[1]{\textcolor[rgb]{0.13,0.29,0.53}{\textbf{#1}}}
\newcommand{\DataTypeTok}[1]{\textcolor[rgb]{0.13,0.29,0.53}{#1}}
\newcommand{\DecValTok}[1]{\textcolor[rgb]{0.00,0.00,0.81}{#1}}
\newcommand{\DocumentationTok}[1]{\textcolor[rgb]{0.56,0.35,0.01}{\textbf{\textit{#1}}}}
\newcommand{\ErrorTok}[1]{\textcolor[rgb]{0.64,0.00,0.00}{\textbf{#1}}}
\newcommand{\ExtensionTok}[1]{#1}
\newcommand{\FloatTok}[1]{\textcolor[rgb]{0.00,0.00,0.81}{#1}}
\newcommand{\FunctionTok}[1]{\textcolor[rgb]{0.00,0.00,0.00}{#1}}
\newcommand{\ImportTok}[1]{#1}
\newcommand{\InformationTok}[1]{\textcolor[rgb]{0.56,0.35,0.01}{\textbf{\textit{#1}}}}
\newcommand{\KeywordTok}[1]{\textcolor[rgb]{0.13,0.29,0.53}{\textbf{#1}}}
\newcommand{\NormalTok}[1]{#1}
\newcommand{\OperatorTok}[1]{\textcolor[rgb]{0.81,0.36,0.00}{\textbf{#1}}}
\newcommand{\OtherTok}[1]{\textcolor[rgb]{0.56,0.35,0.01}{#1}}
\newcommand{\PreprocessorTok}[1]{\textcolor[rgb]{0.56,0.35,0.01}{\textit{#1}}}
\newcommand{\RegionMarkerTok}[1]{#1}
\newcommand{\SpecialCharTok}[1]{\textcolor[rgb]{0.00,0.00,0.00}{#1}}
\newcommand{\SpecialStringTok}[1]{\textcolor[rgb]{0.31,0.60,0.02}{#1}}
\newcommand{\StringTok}[1]{\textcolor[rgb]{0.31,0.60,0.02}{#1}}
\newcommand{\VariableTok}[1]{\textcolor[rgb]{0.00,0.00,0.00}{#1}}
\newcommand{\VerbatimStringTok}[1]{\textcolor[rgb]{0.31,0.60,0.02}{#1}}
\newcommand{\WarningTok}[1]{\textcolor[rgb]{0.56,0.35,0.01}{\textbf{\textit{#1}}}}
\usepackage{graphicx}
\makeatletter
\def\maxwidth{\ifdim\Gin@nat@width>\linewidth\linewidth\else\Gin@nat@width\fi}
\def\maxheight{\ifdim\Gin@nat@height>\textheight\textheight\else\Gin@nat@height\fi}
\makeatother
% Scale images if necessary, so that they will not overflow the page
% margins by default, and it is still possible to overwrite the defaults
% using explicit options in \includegraphics[width, height, ...]{}
\setkeys{Gin}{width=\maxwidth,height=\maxheight,keepaspectratio}
% Set default figure placement to htbp
\makeatletter
\def\fps@figure{htbp}
\makeatother
\setlength{\emergencystretch}{3em} % prevent overfull lines
\providecommand{\tightlist}{%
  \setlength{\itemsep}{0pt}\setlength{\parskip}{0pt}}
\setcounter{secnumdepth}{-\maxdimen} % remove section numbering
\ifLuaTeX
  \usepackage{selnolig}  % disable illegal ligatures
\fi

\title{Spotify Podcasts vs.~Audio Features: Statistical Testing}
\author{Lara Karacasu \& Quinn Booth}
\date{2023-04-23}

\begin{document}
\maketitle

\hypertarget{data-preparation}{%
\subsection{Data Preparation}\label{data-preparation}}

\begin{Shaded}
\begin{Highlighting}[]
\CommentTok{\# Read in CSV}
\NormalTok{data1 }\OtherTok{\textless{}{-}} \FunctionTok{read.csv}\NormalTok{(}\AttributeTok{file =} \StringTok{"final\_dataset.csv"}\NormalTok{, }\AttributeTok{header =}\NormalTok{ T, }\AttributeTok{stringsAsFactors =}\NormalTok{ T)}
\NormalTok{data1 }\OtherTok{\textless{}{-}} \FunctionTok{as\_tibble}\NormalTok{(data1)}

\CommentTok{\# Convert audio feature columns to double type}
\NormalTok{data1 }\OtherTok{\textless{}{-}}\NormalTok{ data1 }\SpecialCharTok{\%\textgreater{}\%} 
  \FunctionTok{mutate\_at}\NormalTok{(}\FunctionTok{c}\NormalTok{(}\DecValTok{13}\SpecialCharTok{:}\DecValTok{100}\NormalTok{), as.character) }\SpecialCharTok{\%\textgreater{}\%}
  \FunctionTok{mutate\_at}\NormalTok{(}\FunctionTok{c}\NormalTok{(}\DecValTok{13}\SpecialCharTok{:}\DecValTok{100}\NormalTok{), as.double)}

\CommentTok{\# Replace empty string factor}
\FunctionTok{levels}\NormalTok{(data1}\SpecialCharTok{$}\NormalTok{apple\_category)[}\DecValTok{1}\NormalTok{] }\OtherTok{\textless{}{-}} \StringTok{"Not specified"}

\CommentTok{\# Print head of data}
\FunctionTok{head}\NormalTok{(data1)}
\end{Highlighting}
\end{Shaded}

\begin{verbatim}
## # A tibble: 6 x 100
##       X show_name      episode_name episode_date rss_link apple_link apple_stars
##   <int> <fct>          <fct>        <fct>        <fct>    <fct>            <dbl>
## 1     0 Kop On! A Liv~ "[\"episode~ ['2-26-2019~ https:/~ https://p~         4.5
## 2     1 2 Massage The~ "['creating~ ['5-14-2019~ https:/~ https://p~         5  
## 3     2 2. Bundesliga~ "['2. bunde~ ['12-04-201~ https:/~ https://p~         4.9
## 4     3 2MinuteTennis~ "['episode ~ ['7-13-2019~ https:/~ https://p~         5  
## 5     4 4th & Gold Po~ "[\"murphy'~ ['9-23-2019~ https:/~ https://p~         4.7
## 6     5 76ers Podcast~ "['rewind v~ ['10-09-201~ https:/~ https://p~         4.6
## # i 93 more variables: apple_ratings <int>, apple_category <fct>,
## #   show_filename_prefix <fct>, episode_filename_prefix <fct>, duration <fct>,
## #   F0semitoneFrom27.5Hz_sma3nz_amean <dbl>,
## #   F0semitoneFrom27.5Hz_sma3nz_stddevNorm <dbl>,
## #   F0semitoneFrom27.5Hz_sma3nz_percentile20.0 <dbl>,
## #   F0semitoneFrom27.5Hz_sma3nz_percentile50.0 <dbl>,
## #   F0semitoneFrom27.5Hz_sma3nz_percentile80.0 <dbl>, ...
\end{verbatim}

\hypertarget{grouping-by-number-of-ratings-for-t-tests}{%
\subsection{Grouping by Number of Ratings (for
t-tests)}\label{grouping-by-number-of-ratings-for-t-tests}}

\begin{Shaded}
\begin{Highlighting}[]
\CommentTok{\#top\_ratings \textless{}{-} quantile(data1$apple\_ratings, 0.5, na.rm = TRUE)}
\CommentTok{\#low\_ratings \textless{}{-} quantile(data1$apple\_ratings, 0.5, na.rm = TRUE)}

\NormalTok{median\_ratings }\OtherTok{=} \FunctionTok{median}\NormalTok{(data1}\SpecialCharTok{$}\NormalTok{apple\_ratings, }\AttributeTok{na.rm =} \ConstantTok{TRUE}\NormalTok{)}

\NormalTok{ratings\_success }\OtherTok{\textless{}{-}} \FunctionTok{as.factor}\NormalTok{(}\FunctionTok{case\_when}\NormalTok{(}
\NormalTok{    data1}\SpecialCharTok{$}\NormalTok{apple\_ratings }\SpecialCharTok{\textless{}}\NormalTok{ median\_ratings }\SpecialCharTok{\textasciitilde{}} \StringTok{"low"}\NormalTok{,}
\NormalTok{    data1}\SpecialCharTok{$}\NormalTok{apple\_ratings }\SpecialCharTok{\textgreater{}}\NormalTok{ median\_ratings }\SpecialCharTok{\textasciitilde{}} \StringTok{"high"}\NormalTok{,}
    \ConstantTok{TRUE} \SpecialCharTok{\textasciitilde{}} \StringTok{"medium"}
\NormalTok{  ))}

\NormalTok{data3 }\OtherTok{\textless{}{-}}\NormalTok{ data1 }\SpecialCharTok{\%\textgreater{}\%}
  \FunctionTok{mutate}\NormalTok{(data1, ratings\_success) }\SpecialCharTok{\%\textgreater{}\%}
  \FunctionTok{relocate}\NormalTok{(ratings\_success, }\AttributeTok{.after =}\NormalTok{ apple\_ratings)}

\NormalTok{data3 }\OtherTok{\textless{}{-}}\NormalTok{ data3 }\SpecialCharTok{\%\textgreater{}\%}
  \FunctionTok{filter}\NormalTok{(ratings\_success }\SpecialCharTok{==} \StringTok{"low"} \SpecialCharTok{|}\NormalTok{ ratings\_success }\SpecialCharTok{==} \StringTok{"high"}\NormalTok{) }\SpecialCharTok{\%\textgreater{}\%}
  \FunctionTok{drop\_na}\NormalTok{()}
\FunctionTok{print}\NormalTok{(}\FunctionTok{count}\NormalTok{(data3))}
\end{Highlighting}
\end{Shaded}

\begin{verbatim}
## # A tibble: 1 x 1
##       n
##   <int>
## 1   398
\end{verbatim}

\hypertarget{grouping-by-star-ratings-for-t-tests}{%
\subsection{Grouping by Star Ratings (for
t-tests)}\label{grouping-by-star-ratings-for-t-tests}}

\begin{Shaded}
\begin{Highlighting}[]
\CommentTok{\#top\_stars \textless{}{-} quantile(data1$apple\_stars, 0.5, na.rm = TRUE)}
\CommentTok{\#low\_stars \textless{}{-} quantile(data1$apple\_stars, 0.5, na.rm = TRUE)}

\NormalTok{median\_stars }\OtherTok{=} \FunctionTok{median}\NormalTok{(data1}\SpecialCharTok{$}\NormalTok{apple\_stars, }\AttributeTok{na.rm =} \ConstantTok{TRUE}\NormalTok{)}

\NormalTok{stars\_success }\OtherTok{\textless{}{-}} \FunctionTok{as.factor}\NormalTok{(}\FunctionTok{case\_when}\NormalTok{(}
\NormalTok{    data1}\SpecialCharTok{$}\NormalTok{apple\_stars }\SpecialCharTok{\textless{}}\NormalTok{ median\_stars }\SpecialCharTok{\textasciitilde{}} \StringTok{"low"}\NormalTok{,}
\NormalTok{    data1}\SpecialCharTok{$}\NormalTok{apple\_stars }\SpecialCharTok{\textgreater{}}\NormalTok{ median\_stars }\SpecialCharTok{\textasciitilde{}} \StringTok{"high"}\NormalTok{,}
    \ConstantTok{TRUE} \SpecialCharTok{\textasciitilde{}} \StringTok{"medium"}
\NormalTok{  ))}

\NormalTok{data4 }\OtherTok{\textless{}{-}}\NormalTok{ data1 }\SpecialCharTok{\%\textgreater{}\%}
  \FunctionTok{mutate}\NormalTok{(data1, stars\_success) }\SpecialCharTok{\%\textgreater{}\%}
  \FunctionTok{relocate}\NormalTok{(stars\_success, }\AttributeTok{.after =}\NormalTok{ apple\_ratings)}

\NormalTok{data4 }\OtherTok{\textless{}{-}}\NormalTok{ data4 }\SpecialCharTok{\%\textgreater{}\%}
  \FunctionTok{filter}\NormalTok{(stars\_success }\SpecialCharTok{==} \StringTok{"low"} \SpecialCharTok{|}\NormalTok{ stars\_success }\SpecialCharTok{==} \StringTok{"high"}\NormalTok{) }\SpecialCharTok{\%\textgreater{}\%}
  \FunctionTok{drop\_na}\NormalTok{()}
\FunctionTok{print}\NormalTok{(}\FunctionTok{count}\NormalTok{(data4))}
\end{Highlighting}
\end{Shaded}

\begin{verbatim}
## # A tibble: 1 x 1
##       n
##   <int>
## 1   326
\end{verbatim}

\hypertarget{statistical-testing}{%
\subsection{Statistical Testing}\label{statistical-testing}}

We are focusing on five audio features: fundamental frequency, jitter,
shimmer, HNR, and loudness. In our dataset, these are modeled by the
following, respectively: F0semitoneFrom27.5Hz\_sma3nz\_amean,
jitterLocal\_sma3nz\_amean, shimmerLocaldB\_sma3nz\_amean,
HNRdBACF\_sma3nz\_amean, and loudness\_sma3\_amean We also have two
independent variables: average rating and average engagement. In our
dataset, these are modeled by the following, respectively: apple\_stars
and apple\_ratings.

We will subset our data to obtain two buckets: high and low success
podcasts. Our podcasts with both star ratings and engagement ratings
above the median of the dataset are considered high success, and
podcasts with both metrics below the median of the dataset are
considered low success.

\hypertarget{hypothesis-1}{%
\subsubsection{Hypothesis 1}\label{hypothesis-1}}

Podcasts with higher star ratings will differ significantly from
podcasts with lower star ratings across numerous acoustic features.

\hypertarget{hypothesis-2}{%
\subsubsection{Hypothesis 2}\label{hypothesis-2}}

Podcasts with more raters will differ significantly from podcasts with
fewer raters across numerous acoustic features.

\hypertarget{testing-against-number-of-ratings}{%
\section{Testing against number of
ratings}\label{testing-against-number-of-ratings}}

\begin{Shaded}
\begin{Highlighting}[]
\CommentTok{\# conduct t{-}tests for each acoustic feature comparing high and low success groups}
\NormalTok{ttest\_fundamental\_frequency }\OtherTok{\textless{}{-}} \FunctionTok{t.test}\NormalTok{(data3}\SpecialCharTok{$}\NormalTok{F0semitoneFrom27}\FloatTok{.5}\NormalTok{Hz\_sma3nz\_amean }\SpecialCharTok{\textasciitilde{}}\NormalTok{ data3}\SpecialCharTok{$}\NormalTok{ratings\_success, }\AttributeTok{alternative =} \StringTok{"two.sided"}\NormalTok{)}
\NormalTok{ttest\_jitter }\OtherTok{\textless{}{-}} \FunctionTok{t.test}\NormalTok{(data3}\SpecialCharTok{$}\NormalTok{jitterLocal\_sma3nz\_amean }\SpecialCharTok{\textasciitilde{}}\NormalTok{ data3}\SpecialCharTok{$}\NormalTok{ratings\_success, }\AttributeTok{alternative =} \StringTok{"two.sided"}\NormalTok{)}
\NormalTok{ttest\_shimmer }\OtherTok{\textless{}{-}} \FunctionTok{t.test}\NormalTok{(data3}\SpecialCharTok{$}\NormalTok{shimmerLocaldB\_sma3nz\_amean }\SpecialCharTok{\textasciitilde{}}\NormalTok{ data3}\SpecialCharTok{$}\NormalTok{ratings\_success, }\AttributeTok{alternative =} \StringTok{"two.sided"}\NormalTok{)}
\NormalTok{ttest\_HNR }\OtherTok{\textless{}{-}} \FunctionTok{t.test}\NormalTok{(data3}\SpecialCharTok{$}\NormalTok{HNRdBACF\_sma3nz\_amean }\SpecialCharTok{\textasciitilde{}}\NormalTok{ data3}\SpecialCharTok{$}\NormalTok{ratings\_success, }\AttributeTok{alternative =} \StringTok{"two.sided"}\NormalTok{)}
\NormalTok{ttest\_loudness }\OtherTok{\textless{}{-}} \FunctionTok{t.test}\NormalTok{(data3}\SpecialCharTok{$}\NormalTok{loudness\_sma3\_amean }\SpecialCharTok{\textasciitilde{}}\NormalTok{ data3}\SpecialCharTok{$}\NormalTok{ratings\_success, }\AttributeTok{alternative =} \StringTok{"two.sided"}\NormalTok{)}

\CommentTok{\# perform Benjamini{-}Hochberg correction}
\NormalTok{pvalues }\OtherTok{\textless{}{-}} \FunctionTok{c}\NormalTok{(ttest\_fundamental\_frequency}\SpecialCharTok{$}\NormalTok{p.value, ttest\_jitter}\SpecialCharTok{$}\NormalTok{p.value, ttest\_shimmer}\SpecialCharTok{$}\NormalTok{p.value, ttest\_HNR}\SpecialCharTok{$}\NormalTok{p.value, ttest\_loudness}\SpecialCharTok{$}\NormalTok{p.value)}
\NormalTok{adjusted\_pvalues }\OtherTok{\textless{}{-}} \FunctionTok{p.adjust}\NormalTok{(pvalues, }\AttributeTok{method =} \StringTok{"BH"}\NormalTok{)}

\CommentTok{\# print the p{-}values and adjusted p{-}values for each t{-}test}
\FunctionTok{print}\NormalTok{(}\FunctionTok{paste}\NormalTok{(}\StringTok{"T{-}test for fundamental frequency: p{-}value ="}\NormalTok{, ttest\_fundamental\_frequency}\SpecialCharTok{$}\NormalTok{p.value, }\StringTok{", adjusted p{-}value ="}\NormalTok{, adjusted\_pvalues[}\DecValTok{1}\NormalTok{]))}
\end{Highlighting}
\end{Shaded}

\begin{verbatim}
## [1] "T-test for fundamental frequency: p-value = 0.0193629494413261 , adjusted p-value = 0.0242036868016576"
\end{verbatim}

\begin{Shaded}
\begin{Highlighting}[]
\FunctionTok{print}\NormalTok{(}\FunctionTok{paste}\NormalTok{(}\StringTok{"T{-}test for jitter: p{-}value ="}\NormalTok{, ttest\_jitter}\SpecialCharTok{$}\NormalTok{p.value, }\StringTok{", adjusted p{-}value ="}\NormalTok{, adjusted\_pvalues[}\DecValTok{2}\NormalTok{]))}
\end{Highlighting}
\end{Shaded}

\begin{verbatim}
## [1] "T-test for jitter: p-value = 0.000661550780491062 , adjusted p-value = 0.00330775390245531"
\end{verbatim}

\begin{Shaded}
\begin{Highlighting}[]
\FunctionTok{print}\NormalTok{(}\FunctionTok{paste}\NormalTok{(}\StringTok{"T{-}test for shimmer: p{-}value ="}\NormalTok{, ttest\_shimmer}\SpecialCharTok{$}\NormalTok{p.value, }\StringTok{", adjusted p{-}value ="}\NormalTok{, adjusted\_pvalues[}\DecValTok{3}\NormalTok{]))}
\end{Highlighting}
\end{Shaded}

\begin{verbatim}
## [1] "T-test for shimmer: p-value = 0.0069267692193665 , adjusted p-value = 0.0115446153656108"
\end{verbatim}

\begin{Shaded}
\begin{Highlighting}[]
\FunctionTok{print}\NormalTok{(}\FunctionTok{paste}\NormalTok{(}\StringTok{"T{-}test for HNR: p{-}value ="}\NormalTok{, ttest\_HNR}\SpecialCharTok{$}\NormalTok{p.value, }\StringTok{", adjusted p{-}value ="}\NormalTok{, adjusted\_pvalues[}\DecValTok{4}\NormalTok{]))}
\end{Highlighting}
\end{Shaded}

\begin{verbatim}
## [1] "T-test for HNR: p-value = 0.00202372956382691 , adjusted p-value = 0.00505932390956728"
\end{verbatim}

\begin{Shaded}
\begin{Highlighting}[]
\FunctionTok{print}\NormalTok{(}\FunctionTok{paste}\NormalTok{(}\StringTok{"T{-}test for loudness: p{-}value ="}\NormalTok{, ttest\_loudness}\SpecialCharTok{$}\NormalTok{p.value, }\StringTok{", adjusted p{-}value ="}\NormalTok{, adjusted\_pvalues[}\DecValTok{5}\NormalTok{]))}
\end{Highlighting}
\end{Shaded}

\begin{verbatim}
## [1] "T-test for loudness: p-value = 0.404544560030309 , adjusted p-value = 0.404544560030309"
\end{verbatim}

\hypertarget{testing-against-stars}{%
\section{Testing against stars}\label{testing-against-stars}}

\begin{Shaded}
\begin{Highlighting}[]
\CommentTok{\# conduct t{-}tests for each acoustic feature comparing high and low success groups}
\NormalTok{ttest\_fundamental\_frequency }\OtherTok{\textless{}{-}} \FunctionTok{t.test}\NormalTok{(data4}\SpecialCharTok{$}\NormalTok{F0semitoneFrom27}\FloatTok{.5}\NormalTok{Hz\_sma3nz\_amean }\SpecialCharTok{\textasciitilde{}}\NormalTok{ data4}\SpecialCharTok{$}\NormalTok{stars\_success, }\AttributeTok{alternative =} \StringTok{"two.sided"}\NormalTok{)}
\NormalTok{ttest\_jitter }\OtherTok{\textless{}{-}} \FunctionTok{t.test}\NormalTok{(data4}\SpecialCharTok{$}\NormalTok{jitterLocal\_sma3nz\_amean }\SpecialCharTok{\textasciitilde{}}\NormalTok{ data4}\SpecialCharTok{$}\NormalTok{stars\_success, }\AttributeTok{alternative =} \StringTok{"two.sided"}\NormalTok{)}
\NormalTok{ttest\_shimmer }\OtherTok{\textless{}{-}} \FunctionTok{t.test}\NormalTok{(data4}\SpecialCharTok{$}\NormalTok{shimmerLocaldB\_sma3nz\_amean }\SpecialCharTok{\textasciitilde{}}\NormalTok{ data4}\SpecialCharTok{$}\NormalTok{stars\_success, }\AttributeTok{alternative =} \StringTok{"two.sided"}\NormalTok{)}
\NormalTok{ttest\_HNR }\OtherTok{\textless{}{-}} \FunctionTok{t.test}\NormalTok{(data4}\SpecialCharTok{$}\NormalTok{HNRdBACF\_sma3nz\_amean }\SpecialCharTok{\textasciitilde{}}\NormalTok{ data4}\SpecialCharTok{$}\NormalTok{stars\_success, }\AttributeTok{alternative =} \StringTok{"two.sided"}\NormalTok{)}
\NormalTok{ttest\_loudness }\OtherTok{\textless{}{-}} \FunctionTok{t.test}\NormalTok{(data4}\SpecialCharTok{$}\NormalTok{loudness\_sma3\_amean }\SpecialCharTok{\textasciitilde{}}\NormalTok{ data4}\SpecialCharTok{$}\NormalTok{stars\_success, }\AttributeTok{alternative =} \StringTok{"two.sided"}\NormalTok{)}

\CommentTok{\# perform Benjamini{-}Hochberg correction}
\NormalTok{pvalues }\OtherTok{\textless{}{-}} \FunctionTok{c}\NormalTok{(ttest\_fundamental\_frequency}\SpecialCharTok{$}\NormalTok{p.value, ttest\_jitter}\SpecialCharTok{$}\NormalTok{p.value, ttest\_shimmer}\SpecialCharTok{$}\NormalTok{p.value, ttest\_HNR}\SpecialCharTok{$}\NormalTok{p.value, ttest\_loudness}\SpecialCharTok{$}\NormalTok{p.value)}
\NormalTok{adjusted\_pvalues }\OtherTok{\textless{}{-}} \FunctionTok{p.adjust}\NormalTok{(pvalues, }\AttributeTok{method =} \StringTok{"BH"}\NormalTok{)}

\CommentTok{\# print the p{-}values and adjusted p{-}values for each t{-}test}
\FunctionTok{print}\NormalTok{(}\FunctionTok{paste}\NormalTok{(}\StringTok{"T{-}test for fundamental frequency: p{-}value ="}\NormalTok{, ttest\_fundamental\_frequency}\SpecialCharTok{$}\NormalTok{p.value, }\StringTok{", adjusted p{-}value ="}\NormalTok{, adjusted\_pvalues[}\DecValTok{1}\NormalTok{]))}
\end{Highlighting}
\end{Shaded}

\begin{verbatim}
## [1] "T-test for fundamental frequency: p-value = 0.63097153372713 , adjusted p-value = 0.788714417158912"
\end{verbatim}

\begin{Shaded}
\begin{Highlighting}[]
\FunctionTok{print}\NormalTok{(}\FunctionTok{paste}\NormalTok{(}\StringTok{"T{-}test for jitter: p{-}value ="}\NormalTok{, ttest\_jitter}\SpecialCharTok{$}\NormalTok{p.value, }\StringTok{", adjusted p{-}value ="}\NormalTok{, adjusted\_pvalues[}\DecValTok{2}\NormalTok{]))}
\end{Highlighting}
\end{Shaded}

\begin{verbatim}
## [1] "T-test for jitter: p-value = 0.189409463009983 , adjusted p-value = 0.473523657524959"
\end{verbatim}

\begin{Shaded}
\begin{Highlighting}[]
\FunctionTok{print}\NormalTok{(}\FunctionTok{paste}\NormalTok{(}\StringTok{"T{-}test for shimmer: p{-}value ="}\NormalTok{, ttest\_shimmer}\SpecialCharTok{$}\NormalTok{p.value, }\StringTok{", adjusted p{-}value ="}\NormalTok{, adjusted\_pvalues[}\DecValTok{3}\NormalTok{]))}
\end{Highlighting}
\end{Shaded}

\begin{verbatim}
## [1] "T-test for shimmer: p-value = 0.29174154963596 , adjusted p-value = 0.486235916059934"
\end{verbatim}

\begin{Shaded}
\begin{Highlighting}[]
\FunctionTok{print}\NormalTok{(}\FunctionTok{paste}\NormalTok{(}\StringTok{"T{-}test for HNR: p{-}value ="}\NormalTok{, ttest\_HNR}\SpecialCharTok{$}\NormalTok{p.value, }\StringTok{", adjusted p{-}value ="}\NormalTok{, adjusted\_pvalues[}\DecValTok{4}\NormalTok{]))}
\end{Highlighting}
\end{Shaded}

\begin{verbatim}
## [1] "T-test for HNR: p-value = 0.947461634283994 , adjusted p-value = 0.947461634283994"
\end{verbatim}

\begin{Shaded}
\begin{Highlighting}[]
\FunctionTok{print}\NormalTok{(}\FunctionTok{paste}\NormalTok{(}\StringTok{"T{-}test for loudness: p{-}value ="}\NormalTok{, ttest\_loudness}\SpecialCharTok{$}\NormalTok{p.value, }\StringTok{", adjusted p{-}value ="}\NormalTok{, adjusted\_pvalues[}\DecValTok{5}\NormalTok{]))}
\end{Highlighting}
\end{Shaded}

\begin{verbatim}
## [1] "T-test for loudness: p-value = 0.0846579934505076 , adjusted p-value = 0.423289967252538"
\end{verbatim}

\hypertarget{hypothesis-3}{%
\subsubsection{Hypothesis 3}\label{hypothesis-3}}

Podcasts with higher ratings and engagement will have significantly:
higher fundamental frequency, lower jitter, lower shimmer, lower HNR,
greater loudness.

To test this hypothesis, we need to compare the median values of each
acoustic feature for podcasts with high versus low ratings and
engagement. Podcasts which have higher star ratings AND a higher number
of raters on Apple Podcasts are considered high success podcasts, while
podcasts with lower star ratings and a lower number of raters on Apple
Podcasts are considered low success podcasts.

We subset the data into a high and low success groups. We then conducts
t-tests for each acoustic feature, comparing the high success group to
the low success group. The resulting p-values are stored in a vector,
and the Benjamini-Hochberg correction is applied using the p.adjust
function. Finally, the p-values with the Benjamini-Hochberg correction
are printed for each feature.

\hypertarget{checking-for-negative-relationship-high-attribute-low-ratingstars}{%
\subsection{Checking for negative relationship (high attribute = low
rating/stars)}\label{checking-for-negative-relationship-high-attribute-low-ratingstars}}

\hypertarget{testing-against-number-of-ratings-1}{%
\section{Testing against number of
ratings}\label{testing-against-number-of-ratings-1}}

\begin{Shaded}
\begin{Highlighting}[]
\CommentTok{\# conduct t{-}tests for each acoustic feature comparing high and low success groups}
\NormalTok{ttest\_fundamental\_frequency }\OtherTok{\textless{}{-}} \FunctionTok{t.test}\NormalTok{(data3}\SpecialCharTok{$}\NormalTok{F0semitoneFrom27}\FloatTok{.5}\NormalTok{Hz\_sma3nz\_amean }\SpecialCharTok{\textasciitilde{}}\NormalTok{ data3}\SpecialCharTok{$}\NormalTok{ratings\_success, }\AttributeTok{alternative =} \StringTok{"less"}\NormalTok{)}
\NormalTok{ttest\_jitter }\OtherTok{\textless{}{-}} \FunctionTok{t.test}\NormalTok{(data3}\SpecialCharTok{$}\NormalTok{jitterLocal\_sma3nz\_amean }\SpecialCharTok{\textasciitilde{}}\NormalTok{ data3}\SpecialCharTok{$}\NormalTok{ratings\_success, }\AttributeTok{alternative =} \StringTok{"less"}\NormalTok{)}
\NormalTok{ttest\_shimmer }\OtherTok{\textless{}{-}} \FunctionTok{t.test}\NormalTok{(data3}\SpecialCharTok{$}\NormalTok{shimmerLocaldB\_sma3nz\_amean }\SpecialCharTok{\textasciitilde{}}\NormalTok{ data3}\SpecialCharTok{$}\NormalTok{ratings\_success, }\AttributeTok{alternative =} \StringTok{"less"}\NormalTok{)}
\NormalTok{ttest\_HNR }\OtherTok{\textless{}{-}} \FunctionTok{t.test}\NormalTok{(data3}\SpecialCharTok{$}\NormalTok{HNRdBACF\_sma3nz\_amean }\SpecialCharTok{\textasciitilde{}}\NormalTok{ data3}\SpecialCharTok{$}\NormalTok{ratings\_success, }\AttributeTok{alternative =} \StringTok{"less"}\NormalTok{)}
\NormalTok{ttest\_loudness }\OtherTok{\textless{}{-}} \FunctionTok{t.test}\NormalTok{(data3}\SpecialCharTok{$}\NormalTok{loudness\_sma3\_amean }\SpecialCharTok{\textasciitilde{}}\NormalTok{ data3}\SpecialCharTok{$}\NormalTok{ratings\_success, }\AttributeTok{alternative =} \StringTok{"less"}\NormalTok{)}

\CommentTok{\# perform Benjamini{-}Hochberg correction}
\NormalTok{pvalues }\OtherTok{\textless{}{-}} \FunctionTok{c}\NormalTok{(ttest\_fundamental\_frequency}\SpecialCharTok{$}\NormalTok{p.value, ttest\_jitter}\SpecialCharTok{$}\NormalTok{p.value, ttest\_shimmer}\SpecialCharTok{$}\NormalTok{p.value, ttest\_HNR}\SpecialCharTok{$}\NormalTok{p.value, ttest\_loudness}\SpecialCharTok{$}\NormalTok{p.value)}
\NormalTok{adjusted\_pvalues }\OtherTok{\textless{}{-}} \FunctionTok{p.adjust}\NormalTok{(pvalues, }\AttributeTok{method =} \StringTok{"BH"}\NormalTok{)}

\CommentTok{\# print the p{-}values and adjusted p{-}values for each t{-}test}
\FunctionTok{print}\NormalTok{(}\FunctionTok{paste}\NormalTok{(}\StringTok{"T{-}test for fundamental frequency: p{-}value ="}\NormalTok{, ttest\_fundamental\_frequency}\SpecialCharTok{$}\NormalTok{p.value, }\StringTok{", adjusted p{-}value ="}\NormalTok{, adjusted\_pvalues[}\DecValTok{1}\NormalTok{]))}
\end{Highlighting}
\end{Shaded}

\begin{verbatim}
## [1] "T-test for fundamental frequency: p-value = 0.990318525279337 , adjusted p-value = 0.998988135218086"
\end{verbatim}

\begin{Shaded}
\begin{Highlighting}[]
\FunctionTok{print}\NormalTok{(}\FunctionTok{paste}\NormalTok{(}\StringTok{"T{-}test for jitter: p{-}value ="}\NormalTok{, ttest\_jitter}\SpecialCharTok{$}\NormalTok{p.value, }\StringTok{", adjusted p{-}value ="}\NormalTok{, adjusted\_pvalues[}\DecValTok{2}\NormalTok{]))}
\end{Highlighting}
\end{Shaded}

\begin{verbatim}
## [1] "T-test for jitter: p-value = 0.000330775390245531 , adjusted p-value = 0.00165387695122766"
\end{verbatim}

\begin{Shaded}
\begin{Highlighting}[]
\FunctionTok{print}\NormalTok{(}\FunctionTok{paste}\NormalTok{(}\StringTok{"T{-}test for shimmer: p{-}value ="}\NormalTok{, ttest\_shimmer}\SpecialCharTok{$}\NormalTok{p.value, }\StringTok{", adjusted p{-}value ="}\NormalTok{, adjusted\_pvalues[}\DecValTok{3}\NormalTok{]))}
\end{Highlighting}
\end{Shaded}

\begin{verbatim}
## [1] "T-test for shimmer: p-value = 0.00346338460968325 , adjusted p-value = 0.00865846152420813"
\end{verbatim}

\begin{Shaded}
\begin{Highlighting}[]
\FunctionTok{print}\NormalTok{(}\FunctionTok{paste}\NormalTok{(}\StringTok{"T{-}test for HNR: p{-}value ="}\NormalTok{, ttest\_HNR}\SpecialCharTok{$}\NormalTok{p.value, }\StringTok{", adjusted p{-}value ="}\NormalTok{, adjusted\_pvalues[}\DecValTok{4}\NormalTok{]))}
\end{Highlighting}
\end{Shaded}

\begin{verbatim}
## [1] "T-test for HNR: p-value = 0.998988135218086 , adjusted p-value = 0.998988135218086"
\end{verbatim}

\begin{Shaded}
\begin{Highlighting}[]
\FunctionTok{print}\NormalTok{(}\FunctionTok{paste}\NormalTok{(}\StringTok{"T{-}test for loudness: p{-}value ="}\NormalTok{, ttest\_loudness}\SpecialCharTok{$}\NormalTok{p.value, }\StringTok{", adjusted p{-}value ="}\NormalTok{, adjusted\_pvalues[}\DecValTok{5}\NormalTok{]))}
\end{Highlighting}
\end{Shaded}

\begin{verbatim}
## [1] "T-test for loudness: p-value = 0.797727719984846 , adjusted p-value = 0.998988135218086"
\end{verbatim}

\begin{Shaded}
\begin{Highlighting}[]
\FunctionTok{print}\NormalTok{(}\FunctionTok{levels}\NormalTok{(data3}\SpecialCharTok{$}\NormalTok{ratings\_success))}
\end{Highlighting}
\end{Shaded}

\begin{verbatim}
## [1] "high"   "low"    "medium"
\end{verbatim}

\hypertarget{testing-against-stars-1}{%
\section{Testing against stars}\label{testing-against-stars-1}}

\begin{Shaded}
\begin{Highlighting}[]
\CommentTok{\# conduct t{-}tests for each acoustic feature comparing high and low success groups}
\NormalTok{ttest\_fundamental\_frequency }\OtherTok{\textless{}{-}} \FunctionTok{t.test}\NormalTok{(data4}\SpecialCharTok{$}\NormalTok{F0semitoneFrom27}\FloatTok{.5}\NormalTok{Hz\_sma3nz\_amean }\SpecialCharTok{\textasciitilde{}}\NormalTok{ data4}\SpecialCharTok{$}\NormalTok{stars\_success, }\AttributeTok{alternative =} \StringTok{"less"}\NormalTok{)}
\NormalTok{ttest\_jitter }\OtherTok{\textless{}{-}} \FunctionTok{t.test}\NormalTok{(data4}\SpecialCharTok{$}\NormalTok{jitterLocal\_sma3nz\_amean }\SpecialCharTok{\textasciitilde{}}\NormalTok{ data4}\SpecialCharTok{$}\NormalTok{stars\_success, }\AttributeTok{alternative =} \StringTok{"less"}\NormalTok{)}
\NormalTok{ttest\_shimmer }\OtherTok{\textless{}{-}} \FunctionTok{t.test}\NormalTok{(data4}\SpecialCharTok{$}\NormalTok{shimmerLocaldB\_sma3nz\_amean }\SpecialCharTok{\textasciitilde{}}\NormalTok{ data4}\SpecialCharTok{$}\NormalTok{stars\_success, }\AttributeTok{alternative =} \StringTok{"less"}\NormalTok{)}
\NormalTok{ttest\_HNR }\OtherTok{\textless{}{-}} \FunctionTok{t.test}\NormalTok{(data4}\SpecialCharTok{$}\NormalTok{HNRdBACF\_sma3nz\_amean }\SpecialCharTok{\textasciitilde{}}\NormalTok{ data4}\SpecialCharTok{$}\NormalTok{stars\_success, }\AttributeTok{alternative =} \StringTok{"less"}\NormalTok{)}
\NormalTok{ttest\_loudness }\OtherTok{\textless{}{-}} \FunctionTok{t.test}\NormalTok{(data4}\SpecialCharTok{$}\NormalTok{loudness\_sma3\_amean }\SpecialCharTok{\textasciitilde{}}\NormalTok{ data4}\SpecialCharTok{$}\NormalTok{stars\_success, }\AttributeTok{alternative =} \StringTok{"less"}\NormalTok{)}

\CommentTok{\# perform Benjamini{-}Hochberg correction}
\NormalTok{pvalues }\OtherTok{\textless{}{-}} \FunctionTok{c}\NormalTok{(ttest\_fundamental\_frequency}\SpecialCharTok{$}\NormalTok{p.value, ttest\_jitter}\SpecialCharTok{$}\NormalTok{p.value, ttest\_shimmer}\SpecialCharTok{$}\NormalTok{p.value, ttest\_HNR}\SpecialCharTok{$}\NormalTok{p.value, ttest\_loudness}\SpecialCharTok{$}\NormalTok{p.value)}
\NormalTok{adjusted\_pvalues }\OtherTok{\textless{}{-}} \FunctionTok{p.adjust}\NormalTok{(pvalues, }\AttributeTok{method =} \StringTok{"BH"}\NormalTok{)}

\CommentTok{\# print the p{-}values and adjusted p{-}values for each t{-}test}
\FunctionTok{print}\NormalTok{(}\FunctionTok{paste}\NormalTok{(}\StringTok{"T{-}test for fundamental frequency: p{-}value ="}\NormalTok{, ttest\_fundamental\_frequency}\SpecialCharTok{$}\NormalTok{p.value, }\StringTok{", adjusted p{-}value ="}\NormalTok{, adjusted\_pvalues[}\DecValTok{1}\NormalTok{]))}
\end{Highlighting}
\end{Shaded}

\begin{verbatim}
## [1] "T-test for fundamental frequency: p-value = 0.684514233136435 , adjusted p-value = 0.957671003274746"
\end{verbatim}

\begin{Shaded}
\begin{Highlighting}[]
\FunctionTok{print}\NormalTok{(}\FunctionTok{paste}\NormalTok{(}\StringTok{"T{-}test for jitter: p{-}value ="}\NormalTok{, ttest\_jitter}\SpecialCharTok{$}\NormalTok{p.value, }\StringTok{", adjusted p{-}value ="}\NormalTok{, adjusted\_pvalues[}\DecValTok{2}\NormalTok{]))}
\end{Highlighting}
\end{Shaded}

\begin{verbatim}
## [1] "T-test for jitter: p-value = 0.905295268495008 , adjusted p-value = 0.957671003274746"
\end{verbatim}

\begin{Shaded}
\begin{Highlighting}[]
\FunctionTok{print}\NormalTok{(}\FunctionTok{paste}\NormalTok{(}\StringTok{"T{-}test for shimmer: p{-}value ="}\NormalTok{, ttest\_shimmer}\SpecialCharTok{$}\NormalTok{p.value, }\StringTok{", adjusted p{-}value ="}\NormalTok{, adjusted\_pvalues[}\DecValTok{3}\NormalTok{]))}
\end{Highlighting}
\end{Shaded}

\begin{verbatim}
## [1] "T-test for shimmer: p-value = 0.85412922518202 , adjusted p-value = 0.957671003274746"
\end{verbatim}

\begin{Shaded}
\begin{Highlighting}[]
\FunctionTok{print}\NormalTok{(}\FunctionTok{paste}\NormalTok{(}\StringTok{"T{-}test for HNR: p{-}value ="}\NormalTok{, ttest\_HNR}\SpecialCharTok{$}\NormalTok{p.value, }\StringTok{", adjusted p{-}value ="}\NormalTok{, adjusted\_pvalues[}\DecValTok{4}\NormalTok{]))}
\end{Highlighting}
\end{Shaded}

\begin{verbatim}
## [1] "T-test for HNR: p-value = 0.473730817141997 , adjusted p-value = 0.957671003274746"
\end{verbatim}

\begin{Shaded}
\begin{Highlighting}[]
\FunctionTok{print}\NormalTok{(}\FunctionTok{paste}\NormalTok{(}\StringTok{"T{-}test for loudness: p{-}value ="}\NormalTok{, ttest\_loudness}\SpecialCharTok{$}\NormalTok{p.value, }\StringTok{", adjusted p{-}value ="}\NormalTok{, adjusted\_pvalues[}\DecValTok{5}\NormalTok{]))}
\end{Highlighting}
\end{Shaded}

\begin{verbatim}
## [1] "T-test for loudness: p-value = 0.957671003274746 , adjusted p-value = 0.957671003274746"
\end{verbatim}

\hypertarget{checking-for-positive-relationship-high-attribute-high-ratingstars}{%
\subsection{Checking for positive relationship (high attribute = high
rating/stars)}\label{checking-for-positive-relationship-high-attribute-high-ratingstars}}

\hypertarget{testing-against-number-of-ratings-2}{%
\section{Testing against number of
ratings}\label{testing-against-number-of-ratings-2}}

\begin{Shaded}
\begin{Highlighting}[]
\CommentTok{\# conduct t{-}tests for each acoustic feature comparing high and low success groups}
\NormalTok{ttest\_fundamental\_frequency }\OtherTok{\textless{}{-}} \FunctionTok{t.test}\NormalTok{(data3}\SpecialCharTok{$}\NormalTok{F0semitoneFrom27}\FloatTok{.5}\NormalTok{Hz\_sma3nz\_amean }\SpecialCharTok{\textasciitilde{}}\NormalTok{ data3}\SpecialCharTok{$}\NormalTok{ratings\_success, }\AttributeTok{alternative =} \StringTok{"greater"}\NormalTok{)}
\NormalTok{ttest\_jitter }\OtherTok{\textless{}{-}} \FunctionTok{t.test}\NormalTok{(data3}\SpecialCharTok{$}\NormalTok{jitterLocal\_sma3nz\_amean }\SpecialCharTok{\textasciitilde{}}\NormalTok{ data3}\SpecialCharTok{$}\NormalTok{ratings\_success, }\AttributeTok{alternative =} \StringTok{"greater"}\NormalTok{)}
\NormalTok{ttest\_shimmer }\OtherTok{\textless{}{-}} \FunctionTok{t.test}\NormalTok{(data3}\SpecialCharTok{$}\NormalTok{shimmerLocaldB\_sma3nz\_amean }\SpecialCharTok{\textasciitilde{}}\NormalTok{ data3}\SpecialCharTok{$}\NormalTok{ratings\_success, }\AttributeTok{alternative =} \StringTok{"greater"}\NormalTok{)}
\NormalTok{ttest\_HNR }\OtherTok{\textless{}{-}} \FunctionTok{t.test}\NormalTok{(data3}\SpecialCharTok{$}\NormalTok{HNRdBACF\_sma3nz\_amean }\SpecialCharTok{\textasciitilde{}}\NormalTok{ data3}\SpecialCharTok{$}\NormalTok{ratings\_success, }\AttributeTok{alternative =} \StringTok{"greater"}\NormalTok{)}
\NormalTok{ttest\_loudness }\OtherTok{\textless{}{-}} \FunctionTok{t.test}\NormalTok{(data3}\SpecialCharTok{$}\NormalTok{loudness\_sma3\_amean }\SpecialCharTok{\textasciitilde{}}\NormalTok{ data3}\SpecialCharTok{$}\NormalTok{ratings\_success, }\AttributeTok{alternative =} \StringTok{"greater"}\NormalTok{)}

\CommentTok{\# perform Benjamini{-}Hochberg correction}
\NormalTok{pvalues }\OtherTok{\textless{}{-}} \FunctionTok{c}\NormalTok{(ttest\_fundamental\_frequency}\SpecialCharTok{$}\NormalTok{p.value, ttest\_jitter}\SpecialCharTok{$}\NormalTok{p.value, ttest\_shimmer}\SpecialCharTok{$}\NormalTok{p.value, ttest\_HNR}\SpecialCharTok{$}\NormalTok{p.value, ttest\_loudness}\SpecialCharTok{$}\NormalTok{p.value)}
\NormalTok{adjusted\_pvalues }\OtherTok{\textless{}{-}} \FunctionTok{p.adjust}\NormalTok{(pvalues, }\AttributeTok{method =} \StringTok{"BH"}\NormalTok{)}

\CommentTok{\# print the p{-}values and adjusted p{-}values for each t{-}test}
\FunctionTok{print}\NormalTok{(}\FunctionTok{paste}\NormalTok{(}\StringTok{"T{-}test for fundamental frequency: p{-}value ="}\NormalTok{, ttest\_fundamental\_frequency}\SpecialCharTok{$}\NormalTok{p.value, }\StringTok{", adjusted p{-}value ="}\NormalTok{, adjusted\_pvalues[}\DecValTok{1}\NormalTok{]))}
\end{Highlighting}
\end{Shaded}

\begin{verbatim}
## [1] "T-test for fundamental frequency: p-value = 0.00968147472066305 , adjusted p-value = 0.0242036868016576"
\end{verbatim}

\begin{Shaded}
\begin{Highlighting}[]
\FunctionTok{print}\NormalTok{(}\FunctionTok{paste}\NormalTok{(}\StringTok{"T{-}test for jitter: p{-}value ="}\NormalTok{, ttest\_jitter}\SpecialCharTok{$}\NormalTok{p.value, }\StringTok{", adjusted p{-}value ="}\NormalTok{, adjusted\_pvalues[}\DecValTok{2}\NormalTok{]))}
\end{Highlighting}
\end{Shaded}

\begin{verbatim}
## [1] "T-test for jitter: p-value = 0.999669224609754 , adjusted p-value = 0.999669224609754"
\end{verbatim}

\begin{Shaded}
\begin{Highlighting}[]
\FunctionTok{print}\NormalTok{(}\FunctionTok{paste}\NormalTok{(}\StringTok{"T{-}test for shimmer: p{-}value ="}\NormalTok{, ttest\_shimmer}\SpecialCharTok{$}\NormalTok{p.value, }\StringTok{", adjusted p{-}value ="}\NormalTok{, adjusted\_pvalues[}\DecValTok{3}\NormalTok{]))}
\end{Highlighting}
\end{Shaded}

\begin{verbatim}
## [1] "T-test for shimmer: p-value = 0.996536615390317 , adjusted p-value = 0.999669224609754"
\end{verbatim}

\begin{Shaded}
\begin{Highlighting}[]
\FunctionTok{print}\NormalTok{(}\FunctionTok{paste}\NormalTok{(}\StringTok{"T{-}test for HNR: p{-}value ="}\NormalTok{, ttest\_HNR}\SpecialCharTok{$}\NormalTok{p.value, }\StringTok{", adjusted p{-}value ="}\NormalTok{, adjusted\_pvalues[}\DecValTok{4}\NormalTok{]))}
\end{Highlighting}
\end{Shaded}

\begin{verbatim}
## [1] "T-test for HNR: p-value = 0.00101186478191346 , adjusted p-value = 0.00505932390956728"
\end{verbatim}

\begin{Shaded}
\begin{Highlighting}[]
\FunctionTok{print}\NormalTok{(}\FunctionTok{paste}\NormalTok{(}\StringTok{"T{-}test for loudness: p{-}value ="}\NormalTok{, ttest\_loudness}\SpecialCharTok{$}\NormalTok{p.value, }\StringTok{", adjusted p{-}value ="}\NormalTok{, adjusted\_pvalues[}\DecValTok{5}\NormalTok{]))}
\end{Highlighting}
\end{Shaded}

\begin{verbatim}
## [1] "T-test for loudness: p-value = 0.202272280015154 , adjusted p-value = 0.337120466691924"
\end{verbatim}

\begin{Shaded}
\begin{Highlighting}[]
\FunctionTok{print}\NormalTok{(}\FunctionTok{levels}\NormalTok{(data3}\SpecialCharTok{$}\NormalTok{ratings\_success))}
\end{Highlighting}
\end{Shaded}

\begin{verbatim}
## [1] "high"   "low"    "medium"
\end{verbatim}

\hypertarget{testing-against-stars-2}{%
\section{Testing against stars}\label{testing-against-stars-2}}

\begin{Shaded}
\begin{Highlighting}[]
\CommentTok{\# conduct t{-}tests for each acoustic feature comparing high and low success groups}
\NormalTok{ttest\_fundamental\_frequency }\OtherTok{\textless{}{-}} \FunctionTok{t.test}\NormalTok{(data4}\SpecialCharTok{$}\NormalTok{F0semitoneFrom27}\FloatTok{.5}\NormalTok{Hz\_sma3nz\_amean }\SpecialCharTok{\textasciitilde{}}\NormalTok{ data4}\SpecialCharTok{$}\NormalTok{stars\_success, }\AttributeTok{alternative =} \StringTok{"greater"}\NormalTok{)}
\NormalTok{ttest\_jitter }\OtherTok{\textless{}{-}} \FunctionTok{t.test}\NormalTok{(data4}\SpecialCharTok{$}\NormalTok{jitterLocal\_sma3nz\_amean }\SpecialCharTok{\textasciitilde{}}\NormalTok{ data4}\SpecialCharTok{$}\NormalTok{stars\_success, }\AttributeTok{alternative =} \StringTok{"greater"}\NormalTok{)}
\NormalTok{ttest\_shimmer }\OtherTok{\textless{}{-}} \FunctionTok{t.test}\NormalTok{(data4}\SpecialCharTok{$}\NormalTok{shimmerLocaldB\_sma3nz\_amean }\SpecialCharTok{\textasciitilde{}}\NormalTok{ data4}\SpecialCharTok{$}\NormalTok{stars\_success, }\AttributeTok{alternative =} \StringTok{"greater"}\NormalTok{)}
\NormalTok{ttest\_HNR }\OtherTok{\textless{}{-}} \FunctionTok{t.test}\NormalTok{(data4}\SpecialCharTok{$}\NormalTok{HNRdBACF\_sma3nz\_amean }\SpecialCharTok{\textasciitilde{}}\NormalTok{ data4}\SpecialCharTok{$}\NormalTok{stars\_success, }\AttributeTok{alternative =} \StringTok{"greater"}\NormalTok{)}
\NormalTok{ttest\_loudness }\OtherTok{\textless{}{-}} \FunctionTok{t.test}\NormalTok{(data4}\SpecialCharTok{$}\NormalTok{loudness\_sma3\_amean }\SpecialCharTok{\textasciitilde{}}\NormalTok{ data4}\SpecialCharTok{$}\NormalTok{stars\_success, }\AttributeTok{alternative =} \StringTok{"greater"}\NormalTok{)}

\CommentTok{\# perform Benjamini{-}Hochberg correction}
\NormalTok{pvalues }\OtherTok{\textless{}{-}} \FunctionTok{c}\NormalTok{(ttest\_fundamental\_frequency}\SpecialCharTok{$}\NormalTok{p.value, ttest\_jitter}\SpecialCharTok{$}\NormalTok{p.value, ttest\_shimmer}\SpecialCharTok{$}\NormalTok{p.value, ttest\_HNR}\SpecialCharTok{$}\NormalTok{p.value, ttest\_loudness}\SpecialCharTok{$}\NormalTok{p.value)}
\NormalTok{adjusted\_pvalues }\OtherTok{\textless{}{-}} \FunctionTok{p.adjust}\NormalTok{(pvalues, }\AttributeTok{method =} \StringTok{"BH"}\NormalTok{)}

\CommentTok{\# print the p{-}values and adjusted p{-}values for each t{-}test}
\FunctionTok{print}\NormalTok{(}\FunctionTok{paste}\NormalTok{(}\StringTok{"T{-}test for fundamental frequency: p{-}value ="}\NormalTok{, ttest\_fundamental\_frequency}\SpecialCharTok{$}\NormalTok{p.value, }\StringTok{", adjusted p{-}value ="}\NormalTok{, adjusted\_pvalues[}\DecValTok{1}\NormalTok{]))}
\end{Highlighting}
\end{Shaded}

\begin{verbatim}
## [1] "T-test for fundamental frequency: p-value = 0.315485766863565 , adjusted p-value = 0.394357208579456"
\end{verbatim}

\begin{Shaded}
\begin{Highlighting}[]
\FunctionTok{print}\NormalTok{(}\FunctionTok{paste}\NormalTok{(}\StringTok{"T{-}test for jitter: p{-}value ="}\NormalTok{, ttest\_jitter}\SpecialCharTok{$}\NormalTok{p.value, }\StringTok{", adjusted p{-}value ="}\NormalTok{, adjusted\_pvalues[}\DecValTok{2}\NormalTok{]))}
\end{Highlighting}
\end{Shaded}

\begin{verbatim}
## [1] "T-test for jitter: p-value = 0.0947047315049917 , adjusted p-value = 0.236761828762479"
\end{verbatim}

\begin{Shaded}
\begin{Highlighting}[]
\FunctionTok{print}\NormalTok{(}\FunctionTok{paste}\NormalTok{(}\StringTok{"T{-}test for shimmer: p{-}value ="}\NormalTok{, ttest\_shimmer}\SpecialCharTok{$}\NormalTok{p.value, }\StringTok{", adjusted p{-}value ="}\NormalTok{, adjusted\_pvalues[}\DecValTok{3}\NormalTok{]))}
\end{Highlighting}
\end{Shaded}

\begin{verbatim}
## [1] "T-test for shimmer: p-value = 0.14587077481798 , adjusted p-value = 0.243117958029967"
\end{verbatim}

\begin{Shaded}
\begin{Highlighting}[]
\FunctionTok{print}\NormalTok{(}\FunctionTok{paste}\NormalTok{(}\StringTok{"T{-}test for HNR: p{-}value ="}\NormalTok{, ttest\_HNR}\SpecialCharTok{$}\NormalTok{p.value, }\StringTok{", adjusted p{-}value ="}\NormalTok{, adjusted\_pvalues[}\DecValTok{4}\NormalTok{]))}
\end{Highlighting}
\end{Shaded}

\begin{verbatim}
## [1] "T-test for HNR: p-value = 0.526269182858003 , adjusted p-value = 0.526269182858003"
\end{verbatim}

\begin{Shaded}
\begin{Highlighting}[]
\FunctionTok{print}\NormalTok{(}\FunctionTok{paste}\NormalTok{(}\StringTok{"T{-}test for loudness: p{-}value ="}\NormalTok{, ttest\_loudness}\SpecialCharTok{$}\NormalTok{p.value, }\StringTok{", adjusted p{-}value ="}\NormalTok{, adjusted\_pvalues[}\DecValTok{5}\NormalTok{]))}
\end{Highlighting}
\end{Shaded}

\begin{verbatim}
## [1] "T-test for loudness: p-value = 0.0423289967252538 , adjusted p-value = 0.211644983626269"
\end{verbatim}

\hypertarget{visualizations-of-the-results}{%
\subsection{Visualizations of the
Results}\label{visualizations-of-the-results}}

\begin{Shaded}
\begin{Highlighting}[]
\CommentTok{\# Fundamental Frequency by Category, by Number of Raters}
\NormalTok{gg3 }\OtherTok{\textless{}{-}} \FunctionTok{ggplot}\NormalTok{(}\AttributeTok{data =}\NormalTok{ data1) }\SpecialCharTok{+}
  \FunctionTok{geom\_point}\NormalTok{(}\AttributeTok{mapping =} \FunctionTok{aes}\NormalTok{(}\AttributeTok{x =}\NormalTok{ F0semitoneFrom27}\FloatTok{.5}\NormalTok{Hz\_sma3nz\_amean, }\AttributeTok{y =}\NormalTok{ apple\_ratings, }\AttributeTok{color =}\NormalTok{ apple\_category)) }\SpecialCharTok{+}
  \FunctionTok{scale\_y\_continuous}\NormalTok{(}\AttributeTok{trans=}\StringTok{\textquotesingle{}log10\textquotesingle{}}\NormalTok{)}
\NormalTok{gg3}
\end{Highlighting}
\end{Shaded}

\includegraphics{statistical_testing_files/figure-latex/unnamed-chunk-10-1.pdf}

\begin{Shaded}
\begin{Highlighting}[]
\CommentTok{\# Jitter by Category, by Number of Raters}

\CommentTok{\#gg3 \textless{}{-} ggplot(data = data1) +}
\CommentTok{\#  geom\_point(mapping = aes(x = jitterLocal\_sma3nz\_amean, y = apple\_ratings, color = apple\_category)) +}
\CommentTok{\#  scale\_y\_continuous(trans=\textquotesingle{}log10\textquotesingle{}) +}
\CommentTok{\#  geom\_smooth(method = \textquotesingle{}lm\textquotesingle{})}
\CommentTok{\#gg3}

\NormalTok{gg3 }\OtherTok{\textless{}{-}} \FunctionTok{ggplot}\NormalTok{(}\AttributeTok{data=}\NormalTok{data1,}\FunctionTok{aes}\NormalTok{(}\AttributeTok{x=}\NormalTok{data1}\SpecialCharTok{$}\NormalTok{jitterLocal\_sma3nz\_amean, }\AttributeTok{y=}\NormalTok{data1}\SpecialCharTok{$}\NormalTok{apple\_ratings)) }\SpecialCharTok{+}
  \FunctionTok{geom\_point}\NormalTok{() }\SpecialCharTok{+} 
  \FunctionTok{geom\_smooth}\NormalTok{(}\AttributeTok{method=}\StringTok{"lm"}\NormalTok{) }\SpecialCharTok{+}
  \FunctionTok{scale\_y\_continuous}\NormalTok{(}\AttributeTok{trans=}\StringTok{\textquotesingle{}log10\textquotesingle{}}\NormalTok{)}
\NormalTok{gg3}
\end{Highlighting}
\end{Shaded}

\includegraphics{statistical_testing_files/figure-latex/unnamed-chunk-11-1.pdf}

\begin{Shaded}
\begin{Highlighting}[]
\CommentTok{\# Shimmer by Category, by Number of Raters}

\CommentTok{\#gg3 \textless{}{-} ggplot(data = data1) +}
\CommentTok{\#  geom\_point(mapping = aes(x = shimmerLocaldB\_sma3nz\_amean, y = apple\_ratings, color = apple\_category)) + }
\CommentTok{\#  scale\_y\_continuous(trans=\textquotesingle{}log10\textquotesingle{})}
\CommentTok{\#gg3}

\NormalTok{gg3 }\OtherTok{\textless{}{-}} \FunctionTok{ggplot}\NormalTok{(}\AttributeTok{data=}\NormalTok{data1,}\FunctionTok{aes}\NormalTok{(}\AttributeTok{x=}\NormalTok{data1}\SpecialCharTok{$}\NormalTok{shimmerLocaldB\_sma3nz\_amean, }\AttributeTok{y=}\NormalTok{data1}\SpecialCharTok{$}\NormalTok{apple\_ratings)) }\SpecialCharTok{+}
  \FunctionTok{geom\_point}\NormalTok{() }\SpecialCharTok{+} 
  \FunctionTok{geom\_smooth}\NormalTok{(}\AttributeTok{method=}\StringTok{"lm"}\NormalTok{) }\SpecialCharTok{+}
  \FunctionTok{scale\_y\_continuous}\NormalTok{(}\AttributeTok{trans=}\StringTok{\textquotesingle{}log10\textquotesingle{}}\NormalTok{)}
\NormalTok{gg3}
\end{Highlighting}
\end{Shaded}

\includegraphics{statistical_testing_files/figure-latex/unnamed-chunk-12-1.pdf}

\begin{Shaded}
\begin{Highlighting}[]
\CommentTok{\# HNR by Category, by Number of Raters}

\CommentTok{\#gg3 \textless{}{-} ggplot(data = data1) +}
\CommentTok{\#  geom\_point(mapping = aes(x = HNRdBACF\_sma3nz\_amean, y = apple\_ratings, color = apple\_category)) +}
\CommentTok{\#  scale\_y\_continuous(trans=\textquotesingle{}log10\textquotesingle{})}
\CommentTok{\#gg3}

\NormalTok{gg3 }\OtherTok{\textless{}{-}} \FunctionTok{ggplot}\NormalTok{(}\AttributeTok{data=}\NormalTok{data1,}\FunctionTok{aes}\NormalTok{(}\AttributeTok{x=}\NormalTok{data1}\SpecialCharTok{$}\NormalTok{HNRdBACF\_sma3nz\_amean, }\AttributeTok{y=}\NormalTok{data1}\SpecialCharTok{$}\NormalTok{apple\_ratings)) }\SpecialCharTok{+}
  \FunctionTok{geom\_point}\NormalTok{() }\SpecialCharTok{+} 
  \FunctionTok{geom\_smooth}\NormalTok{(}\AttributeTok{method=}\StringTok{"lm"}\NormalTok{) }\SpecialCharTok{+}
  \FunctionTok{scale\_y\_continuous}\NormalTok{(}\AttributeTok{trans=}\StringTok{\textquotesingle{}log10\textquotesingle{}}\NormalTok{)}
\NormalTok{gg3}
\end{Highlighting}
\end{Shaded}

\includegraphics{statistical_testing_files/figure-latex/unnamed-chunk-13-1.pdf}

\begin{Shaded}
\begin{Highlighting}[]
\CommentTok{\# Loudness by Category, by Number of Raters}
\NormalTok{gg3 }\OtherTok{\textless{}{-}} \FunctionTok{ggplot}\NormalTok{(}\AttributeTok{data =}\NormalTok{ data1) }\SpecialCharTok{+}
  \FunctionTok{geom\_point}\NormalTok{(}\AttributeTok{mapping =} \FunctionTok{aes}\NormalTok{(}\AttributeTok{x =}\NormalTok{ loudness\_sma3\_amean, }\AttributeTok{y =}\NormalTok{ apple\_ratings, }\AttributeTok{color =}\NormalTok{ apple\_category)) }\SpecialCharTok{+}
  \FunctionTok{scale\_y\_continuous}\NormalTok{(}\AttributeTok{trans=}\StringTok{\textquotesingle{}log10\textquotesingle{}}\NormalTok{)}
\NormalTok{gg3}
\end{Highlighting}
\end{Shaded}

\includegraphics{statistical_testing_files/figure-latex/unnamed-chunk-14-1.pdf}

\hypertarget{misc.-visualizing-the-dataset}{%
\subsection{Misc. Visualizing the
Dataset}\label{misc.-visualizing-the-dataset}}

\begin{Shaded}
\begin{Highlighting}[]
\CommentTok{\# Apple Stars by Category, By Proportion}
\NormalTok{gg1 }\OtherTok{\textless{}{-}} \FunctionTok{ggplot}\NormalTok{(}\AttributeTok{data =}\NormalTok{ data1) }\SpecialCharTok{+}
  \FunctionTok{geom\_bar}\NormalTok{(}\AttributeTok{mapping =} \FunctionTok{aes}\NormalTok{(}\AttributeTok{x =}\NormalTok{ apple\_stars, }\AttributeTok{fill =}\NormalTok{ apple\_category), }\AttributeTok{position =} \StringTok{"fill"}\NormalTok{)}
\NormalTok{gg1}
\end{Highlighting}
\end{Shaded}

\includegraphics{statistical_testing_files/figure-latex/unnamed-chunk-15-1.pdf}

\begin{Shaded}
\begin{Highlighting}[]
\CommentTok{\# Apple Stars by Category, by Count}
\NormalTok{gg2 }\OtherTok{\textless{}{-}} \FunctionTok{ggplot}\NormalTok{(}\AttributeTok{data =}\NormalTok{ data1) }\SpecialCharTok{+}
  \FunctionTok{geom\_bar}\NormalTok{(}\AttributeTok{mapping =} \FunctionTok{aes}\NormalTok{(}\AttributeTok{x =}\NormalTok{ apple\_stars, }\AttributeTok{fill =}\NormalTok{ apple\_category)) }\SpecialCharTok{+}
  \FunctionTok{labs}\NormalTok{(}\AttributeTok{title =} \StringTok{"Star Ratings by Podcast Genre"}\NormalTok{,}
       \AttributeTok{subtitle =} \StringTok{"Data from podcasts.apple.com."}\NormalTok{,}
       \AttributeTok{x =} \StringTok{"Average Apple Star Ranking"}\NormalTok{,}
       \AttributeTok{y =} \StringTok{"Count"}\NormalTok{) }\SpecialCharTok{+}
  \FunctionTok{theme}\NormalTok{(}\AttributeTok{plot.title=}\FunctionTok{element\_text}\NormalTok{(}\AttributeTok{face =} \StringTok{"bold"}\NormalTok{)) }\SpecialCharTok{+}
   \FunctionTok{scale\_fill\_discrete}\NormalTok{(}\AttributeTok{name =} \StringTok{"Genre"}\NormalTok{)}
\NormalTok{gg2}
\end{Highlighting}
\end{Shaded}

\includegraphics{statistical_testing_files/figure-latex/unnamed-chunk-16-1.pdf}

\begin{Shaded}
\begin{Highlighting}[]
\CommentTok{\# Apple Stars by Category, by Number of Raters}
\NormalTok{gg3 }\OtherTok{\textless{}{-}} \FunctionTok{ggplot}\NormalTok{(}\AttributeTok{data =}\NormalTok{ data1) }\SpecialCharTok{+}
  \FunctionTok{geom\_point}\NormalTok{(}\AttributeTok{mapping =} \FunctionTok{aes}\NormalTok{(}\AttributeTok{x =}\NormalTok{ apple\_stars, }\AttributeTok{y =}\NormalTok{ apple\_ratings, }\AttributeTok{color =}\NormalTok{ apple\_category)) }\SpecialCharTok{+}
  \FunctionTok{scale\_y\_continuous}\NormalTok{(}\FunctionTok{c}\NormalTok{(}\DecValTok{0}\NormalTok{, }\DecValTok{30000}\NormalTok{, }\DecValTok{1000}\NormalTok{))}
\NormalTok{gg3}
\end{Highlighting}
\end{Shaded}

\includegraphics{statistical_testing_files/figure-latex/unnamed-chunk-17-1.pdf}

\begin{Shaded}
\begin{Highlighting}[]
\CommentTok{\# Bubble Plot of Stars vs. Raters vs. Category}
\NormalTok{gg\_b1 }\OtherTok{\textless{}{-}} \FunctionTok{ggplot}\NormalTok{(}\AttributeTok{data =}\NormalTok{ data1, }\FunctionTok{aes}\NormalTok{(}\AttributeTok{x =}\NormalTok{ apple\_stars, }\AttributeTok{y =}\NormalTok{ F0semitoneFrom27}\FloatTok{.5}\NormalTok{Hz\_sma3nz\_amean, }\AttributeTok{size =}\NormalTok{ apple\_ratings, }\AttributeTok{color =}\NormalTok{ apple\_category)) }\SpecialCharTok{+}
  \FunctionTok{geom\_point}\NormalTok{()}
\NormalTok{gg\_b1}
\end{Highlighting}
\end{Shaded}

\includegraphics{statistical_testing_files/figure-latex/unnamed-chunk-18-1.pdf}

\begin{Shaded}
\begin{Highlighting}[]
\CommentTok{\# Bubble Plot of Stars vs. Raters vs. Category}
\NormalTok{gg\_b2 }\OtherTok{\textless{}{-}} \FunctionTok{ggplot}\NormalTok{(}\AttributeTok{data =}\NormalTok{ data1, }\FunctionTok{aes}\NormalTok{(}\AttributeTok{x =}\NormalTok{ apple\_stars, }\AttributeTok{y =}\NormalTok{ jitterLocal\_sma3nz\_amean, }\AttributeTok{size =}\NormalTok{ apple\_ratings, }\AttributeTok{color =}\NormalTok{ apple\_category)) }\SpecialCharTok{+}
  \FunctionTok{geom\_point}\NormalTok{()}
\NormalTok{gg\_b2}
\end{Highlighting}
\end{Shaded}

\includegraphics{statistical_testing_files/figure-latex/unnamed-chunk-19-1.pdf}

\begin{Shaded}
\begin{Highlighting}[]
\CommentTok{\# Bubble Plot of Stars vs. Raters vs. Category}
\NormalTok{gg\_b3 }\OtherTok{\textless{}{-}} \FunctionTok{ggplot}\NormalTok{(}\AttributeTok{data =}\NormalTok{ data1, }\FunctionTok{aes}\NormalTok{(}\AttributeTok{x =}\NormalTok{ apple\_stars, }\AttributeTok{y =}\NormalTok{ loudness\_sma3\_amean, }\AttributeTok{size =}\NormalTok{ apple\_ratings, }\AttributeTok{color =}\NormalTok{ apple\_category)) }\SpecialCharTok{+}
  \FunctionTok{geom\_point}\NormalTok{()}
\NormalTok{gg\_b3}
\end{Highlighting}
\end{Shaded}

\includegraphics{statistical_testing_files/figure-latex/unnamed-chunk-20-1.pdf}

\begin{Shaded}
\begin{Highlighting}[]
\CommentTok{\# Bubble Plot of Stars vs. Raters vs. Category}
\NormalTok{gg\_b4 }\OtherTok{\textless{}{-}} \FunctionTok{ggplot}\NormalTok{(}\AttributeTok{data =}\NormalTok{ data1, }\FunctionTok{aes}\NormalTok{(}\AttributeTok{x =}\NormalTok{ apple\_stars, }\AttributeTok{y =}\NormalTok{ loudness\_sma3\_stddevNorm, }\AttributeTok{size =}\NormalTok{ apple\_ratings, }\AttributeTok{color =}\NormalTok{ apple\_category)) }\SpecialCharTok{+}
  \FunctionTok{geom\_point}\NormalTok{()}
\NormalTok{gg\_b4}
\end{Highlighting}
\end{Shaded}

\includegraphics{statistical_testing_files/figure-latex/unnamed-chunk-21-1.pdf}

\begin{Shaded}
\begin{Highlighting}[]
\CommentTok{\# Bubble Plot of Stars vs. Raters vs. Category}
\NormalTok{gg\_b5 }\OtherTok{\textless{}{-}} \FunctionTok{ggplot}\NormalTok{(}\AttributeTok{data =}\NormalTok{ data1, }\FunctionTok{aes}\NormalTok{(}\AttributeTok{x =}\NormalTok{ apple\_stars, }\AttributeTok{y =}\NormalTok{ VoicedSegmentsPerSec, }\AttributeTok{size =}\NormalTok{ apple\_ratings, }\AttributeTok{color =}\NormalTok{ apple\_category)) }\SpecialCharTok{+}
  \FunctionTok{geom\_point}\NormalTok{()}
\NormalTok{gg\_b5}
\end{Highlighting}
\end{Shaded}

\includegraphics{statistical_testing_files/figure-latex/unnamed-chunk-22-1.pdf}

\begin{Shaded}
\begin{Highlighting}[]
\CommentTok{\# Bubble Plot of Raters vs. Stars vs. Category}
\NormalTok{gg\_r1 }\OtherTok{\textless{}{-}} \FunctionTok{ggplot}\NormalTok{(}\AttributeTok{data =}\NormalTok{ data1, }\FunctionTok{aes}\NormalTok{(}\AttributeTok{x =}\NormalTok{ apple\_ratings, }\AttributeTok{y =}\NormalTok{ F0semitoneFrom27}\FloatTok{.5}\NormalTok{Hz\_sma3nz\_amean, }\AttributeTok{color =}\NormalTok{ apple\_category)) }\SpecialCharTok{+}
  \FunctionTok{geom\_point}\NormalTok{() }\SpecialCharTok{+}
    \FunctionTok{scale\_x\_continuous}\NormalTok{(}\AttributeTok{trans=}\StringTok{\textquotesingle{}log10\textquotesingle{}}\NormalTok{)}
\NormalTok{gg\_r1}
\end{Highlighting}
\end{Shaded}

\includegraphics{statistical_testing_files/figure-latex/unnamed-chunk-23-1.pdf}

\begin{Shaded}
\begin{Highlighting}[]
\CommentTok{\# Facet Wrap of Stars vs. Raters vs. Category for features}

\NormalTok{gg5 }\OtherTok{\textless{}{-}} \FunctionTok{ggplot}\NormalTok{(}\AttributeTok{data =}\NormalTok{ data1) }\SpecialCharTok{+}
  \FunctionTok{geom\_point}\NormalTok{(}\AttributeTok{mapping =} \FunctionTok{aes}\NormalTok{(}\AttributeTok{x =}\NormalTok{ apple\_stars, }\AttributeTok{y =}\NormalTok{ F0semitoneFrom27}\FloatTok{.5}\NormalTok{Hz\_sma3nz\_amean, }\AttributeTok{size =}\NormalTok{ apple\_ratings)) }\SpecialCharTok{+}
  \FunctionTok{facet\_wrap}\NormalTok{(}\SpecialCharTok{\textasciitilde{}}\NormalTok{ apple\_category) }\SpecialCharTok{+}
  \FunctionTok{labs}\NormalTok{(}\AttributeTok{title =} \StringTok{"Pitch vs. Star Ratings"}\NormalTok{,}
       \AttributeTok{subtitle =} \StringTok{"Data from podcasts.apple.com; Spotify Podcasts Dataset."}\NormalTok{,}
       \AttributeTok{x =} \StringTok{"Average Apple Star Ranking"}\NormalTok{,}
       \AttributeTok{y =} \StringTok{"Count"}\NormalTok{) }\SpecialCharTok{+}
  \FunctionTok{theme}\NormalTok{(}\AttributeTok{plot.title=}\FunctionTok{element\_text}\NormalTok{(}\AttributeTok{face =} \StringTok{"bold"}\NormalTok{)) }\SpecialCharTok{+}
   \FunctionTok{scale\_fill\_discrete}\NormalTok{(}\AttributeTok{name =} \StringTok{"Number of Raters"}\NormalTok{)}
\NormalTok{gg5}
\end{Highlighting}
\end{Shaded}

\includegraphics{statistical_testing_files/figure-latex/unnamed-chunk-24-1.pdf}

\begin{Shaded}
\begin{Highlighting}[]
\CommentTok{\# Facet Wrap of Stars vs. Raters vs. Category for features}

\NormalTok{gg5 }\OtherTok{\textless{}{-}} \FunctionTok{ggplot}\NormalTok{(}\AttributeTok{data =}\NormalTok{ data1) }\SpecialCharTok{+}
  \FunctionTok{geom\_point}\NormalTok{(}\AttributeTok{mapping =} \FunctionTok{aes}\NormalTok{(}\AttributeTok{x =}\NormalTok{ apple\_stars, }\AttributeTok{y =}\NormalTok{ jitterLocal\_sma3nz\_amean, }\AttributeTok{size =}\NormalTok{ apple\_ratings)) }\SpecialCharTok{+}
  \FunctionTok{facet\_wrap}\NormalTok{(}\SpecialCharTok{\textasciitilde{}}\NormalTok{ apple\_category)}
\NormalTok{gg5}
\end{Highlighting}
\end{Shaded}

\includegraphics{statistical_testing_files/figure-latex/unnamed-chunk-25-1.pdf}

\begin{Shaded}
\begin{Highlighting}[]
\CommentTok{\# Facet Wrap of Stars vs. Raters vs. Category for features}

\NormalTok{gg5 }\OtherTok{\textless{}{-}} \FunctionTok{ggplot}\NormalTok{(}\AttributeTok{data =} \FunctionTok{na.omit}\NormalTok{(data1)) }\SpecialCharTok{+}
  \FunctionTok{geom\_point}\NormalTok{(}\AttributeTok{mapping =} \FunctionTok{aes}\NormalTok{(}\AttributeTok{x =}\NormalTok{ apple\_stars, }\AttributeTok{y =}\NormalTok{ loudness\_sma3\_amean, }\AttributeTok{size =}\NormalTok{ apple\_ratings)) }\SpecialCharTok{+}
  \FunctionTok{facet\_wrap}\NormalTok{(}\SpecialCharTok{\textasciitilde{}}\NormalTok{ apple\_category)}
\NormalTok{gg5}
\end{Highlighting}
\end{Shaded}

\includegraphics{statistical_testing_files/figure-latex/unnamed-chunk-26-1.pdf}

\begin{Shaded}
\begin{Highlighting}[]
\CommentTok{\# Facet Wrap of Stars vs. Raters vs. Category for features}

\NormalTok{gg5 }\OtherTok{\textless{}{-}} \FunctionTok{ggplot}\NormalTok{(}\AttributeTok{data =} \FunctionTok{na.omit}\NormalTok{(data1)) }\SpecialCharTok{+}
  \FunctionTok{geom\_point}\NormalTok{(}\AttributeTok{mapping =} \FunctionTok{aes}\NormalTok{(}\AttributeTok{x =}\NormalTok{ apple\_stars, }\AttributeTok{y =}\NormalTok{ loudness\_sma3\_stddevNorm, }\AttributeTok{size =}\NormalTok{ apple\_ratings)) }\SpecialCharTok{+}
  \FunctionTok{facet\_wrap}\NormalTok{(}\SpecialCharTok{\textasciitilde{}}\NormalTok{ apple\_category)}
\NormalTok{gg5}
\end{Highlighting}
\end{Shaded}

\includegraphics{statistical_testing_files/figure-latex/unnamed-chunk-27-1.pdf}

\begin{Shaded}
\begin{Highlighting}[]
\CommentTok{\# Facet Wrap of Stars vs. Raters vs. Category for features}

\NormalTok{gg5 }\OtherTok{\textless{}{-}} \FunctionTok{ggplot}\NormalTok{(}\AttributeTok{data =} \FunctionTok{na.omit}\NormalTok{(data1)) }\SpecialCharTok{+}
  \FunctionTok{geom\_point}\NormalTok{(}\AttributeTok{mapping =} \FunctionTok{aes}\NormalTok{(}\AttributeTok{x =}\NormalTok{ apple\_stars, }\AttributeTok{y =}\NormalTok{ VoicedSegmentsPerSec, }\AttributeTok{size =}\NormalTok{ apple\_ratings)) }\SpecialCharTok{+}
  \FunctionTok{facet\_wrap}\NormalTok{(}\SpecialCharTok{\textasciitilde{}}\NormalTok{ apple\_category)}
\NormalTok{gg5}
\end{Highlighting}
\end{Shaded}

\includegraphics{statistical_testing_files/figure-latex/unnamed-chunk-28-1.pdf}

\end{document}
